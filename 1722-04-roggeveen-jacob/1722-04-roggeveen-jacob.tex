\documentclass{article}
        \usepackage[utf8]{inputenc}
        \usepackage{xcolor}
        \usepackage[colorlinks=true, allcolors=darkgray]{hyperref}
        \usepackage{chngcntr}% http://ctan.org/pkg/chngcntr
        \counterwithout{subsubsection}{subsection}
        \renewcommand\thesubsection{\Roman{subsection}.}
        \renewcommand\thesubsubsection{§ \arabic{subsubsection}}
        \renewcommand*\contentsname{\centering{Sommaire}}
        
        \title{Découverte de l'Île de Paques\footnote{Ce texte, originellement publié dans \textit{} n°, Oxford, 1970, pp.-, a été encodé dans un document XML puis transformé en fichier \LaTeX{} grâce à une feuille de transformation XSL, par Jean-Damien Généro, ingénieur d'études du CNRS affecté au Centre de recherches historiques (UMR 8558/ CNRS-EHESS).}}
        \author{Jacob Roggeveen}
        \date{Valparaiso, 9 février 1865}
        
        \begin{document}
        \maketitle
        
      
         
        
    Notre latitude estimée était de 27° 4' sud et notre longitude de 266° 31' mn. Notre direction était ouest-1/2 sud et notre vitesse de 7 miles. Nous avions un vent assez fort de nord-nord-ouest et sud-ouest, puis calme, avec un temps gris et de fortes pluies…
         
        
    Après une dizaine d'observations à la lunette, l'\textit{Africaansch Galey} qui naviguait en tête indiqua avoir vu une tortue, de la verdure et au loin des oiseaux. Il nous attendit et signala une île au loin. Il avait vu très distinctement, devant le bâbord droit, une île basse et plate, s'étendant à l'ouest à environ 51/2 miles du navire. Il se dirigea vers elle et navigua avec une petite voile ; il observa de nouveau et se laissa dériver en attendant le lever du jour. Ceci étant, nous demandâmes au capitaine Bouman, qui était en arrière, de noter que nous appellerons cette île « Ile de Pâques » parce qu'elle était découverte par nous le jour de Pâques. Il y avait beaucoup de joie et d'espoir parmi l'équipage, car nous espérions que cette terre basse était l'annonce d'une côte inconnue au sud\footnote{Paasch (Paas) signifie Pâques. L'île était l'île de Pâques, à l'est de la Polynésie. Le sommet le plus élevé est de 1969 pieds. Latitude 27°5' S - longitude 268° 16' à l'est de Ténériffe.}.
         
        
    Ayant une légère brise du sud-est et est-sud-est, la terre étant à l'ouest à 8-9 miles de nous, nous naviguions de l'ouest vers le sud au nord-ouest, ainsi nous allions 10 miles à l'ouest, la position estimée étant 27° 4' de latitude sud et la longitude 265°42'.
         
        
    Durant la 9e observation, l'après-midi, nous avons vu au loin une fumée s'élever en plusieurs endroits ; nous en avons conclu que cette île était habitée. Nous avons informé les capitaines des autres bateaux en précisant qu'il n'était pas nécessaire d'entreprendre un débarquement important pour connaître l'île. Nous avons décidé que 2 chaloupes des bateaux \textit{Arend} et \textit{Thienhoven}, bien équipées et armées, iraient à terre et chercheraient un endroit convenable pour débarquer et sonder les fonds. Cette décision étant prise, nous resterions sur nos bateaux pour la nuit et les commandants des 3 navires de la Compagnie tiendraient à bord du navire \textit{Arend} un conseil, en présence d'un officier secrétaire.
         
            
        \section*{Lundi 6 avril 1722}
    
            
        
    Le responsable de l'expédition rechercha comment aller dans l'île qui se trouvait à environ 2 miles devant nous. Une partie des membres de l'expédition, ayant vu une fumée s'élevant de différents endroits, avait pensé avec raison que cette île paraissant sableuse et stérile était néanmoins habitée.
            
        
    Le responsable de l'expédition ne voulant pas être coupable ou accusé de négligence, décida que nous irions à terre avec deux chaloupes convenablement armées pouvant se défendre dans le cas d'un accueil hostile. Nous ne voulions pas être agressifs, mais connaître les coutumes des habitants, voir ce qu'ils portaient ainsi que leurs ornements et enfin leur donner des graines, des fruits, quelques animaux et rechercher ce que nous pourrions éventuellement échanger. Après que ces considérations furent approuvées, nous avons décidé que deux chaloupes des navires \textit{Arend} et \textit{Thienhoven} iraient dans l'île dès le lever du jour, et que le navire \textit{Africaansche Galey} assurerait leur protection, s'il en était besoin, en suivant les chaloupes le plus près possible à distance raisonnable. Ces décisions furent enregistrées et signées par Jacob Roggeveen, Jan Koster, Cornélius Bouman, Roclof Rosendaal.
            
        
    Le temps étant très mauvais et instable avec de l'orage, une forte pluie et un vent de nord-ouest, le débarquement sur l'île fut retardé. Le matin suivant le capitaine Bouman vit, venant de l'île et se dirigeant vers son bateau, une barque avec un homme complètement nu, ne portant rien sur lui. Celui-ci paraissait très heureux de nous voir et admirait nos bateaux. Il observait la grande hauteur des mâts, la finesse des cordages, les voiles, le canon, il regardait tout avec attention. Quand il se vit dans un miroir, il bougea la tête et regarda au dos du miroir en cherchant la raison de cette vision. Après que nous eûmes ri de son étonnement, nous le renvoyâmes à la côte en lui ayant donné et mis autour du cou un petit miroir, une paire de ciseaux et différentes petites choses avec lesquelles il semblait trouver plaisir et satisfaction\footnote{Bouman, dans son rapport du 7 avril dit que « le visiteur était un homme d'une cinquantaine d'années de peau brune, très fort physiquement, avec une barbiche comme en ont les Turcs. Il était très étonné de voir la conception du bateau et tout ce qu'il y avait à bord. Comme nous ne pouvions pas nous comprendre, nous observions son expression. Nous lui donnâmes un petit miroir, il se regarda et fut effrayé, de même avec le son de la cloche. Nous lui donnâmes un verre de brandy qu'il but et quand il en sentit l'effet, il ouvrit de grands yeux… Nous lui donnâmes un second verre avec un biscuit. Il vit que nous étions habillés, il eut honte de sa nudité, il alla vers une table, il posa ses bras et sa tête puis paraissant parler à son dieu, il leva la tête et les mains plusieurs fois vers le ciel puis parla d'une voix forte pendant une demie-heure et se mit à danser et à chanter. Il dansa avec les marins, un violon joua pour lui et il ne se montra pas étonné. Il se montra très heureux. Nous lui nouâmes sur le corps une pièce d'habit de marin ce qui le rendit joyeux. Son petit bateau était fait de quelques pièces de bois attachés par des liens à deux poutres légères. Il était si léger qu'un homme seul pouvait aisément le porter. C'était pour nous un étonnement de voir qu'un homme seul ait eu l'audace d'aller si loin en mer. Nous étions à trois miles de la côte et nous ne pouvions l'aider ». Behrens (op. cit., I, p. 124 –5) dit que « le visiteur criait d'une voix forte vers la côte, «  Odorroga ! Odorroga ! », semblant implorer son Dieu. Il n'est pas impossible que ce fut une invocation à « Torico », grande idole à l'île de Pâques. Ce fait est reporté dans Kort en nauwkeurig verhaaal et Tweejarige reyze Ces deux documents indiquent que « Dago » était le nom d'une petite idole. Tweejarige reyze, p. 52.}.
            
        
    En approchant à peu de distance de cette terre, nous vîmes clairement que la description d'une île basse et sableuse (d'après, d'une part le capitaine William Dampier, se rapportant au compte-rendu du capitaine Davis, et d'autre part le journal personnel de Lionel Wafer, auquel Dampier avait donné, en l'imprimant et en l'illustrant avec tous ses voyages sur mer, une renommée mondiale…), que Dampier considérait comme étant une pointe du Continent Austral, n'était pas le moins du monde semblable à nos observations : l'île était une succession de hautes terres.
            
        
    Cette Ile de Pâques ne pouvait être cette île sableuse, petite et basse, alors qu'elle avait une circonférence comprise entre 15 et 16 miles et possédait aux extrémités est et ouest, distantes d'environ 5 miles, deux hautes collines qui descendaient graduellement. À la jonction de celle-ci avec la plaine, on voyait trois ou quatre petites élévations de terrain. Etant assez éloignés de l'île, nous avions observé une terre brûlée, une végétation si pauvre que nous pensâmes qu'elle aurait pu être comparée à une île « sableuse ». Partant de ces observations, il fut facile de penser qu'en dehors de cette Ile de Pâques, le but de notre expédition, se trouvait plus à l'est. Les descriptions écrites et orales pouvaient aisément expliquer l'erreur d'appréciation. 
            
        
    Nous avions un vent sud, sud-est, sud-sud-ouest avec des rafales.
            
        
    Après le service du petit déjeuner, nous envoyâmes notre chaloupe bien équipée et armée, comme celle du bateau \textit{Thienhoven}. Après qu'elles se soient approchées de la côte, il fut rapporté que les habitants portaient peu d'habits ou de toutes sortes et de toutes couleurs. Ils nous faisaient signe de venir vers eux, mais les instructions données étaient contraires, et le nombre des indiens présents trop important. Certains pensaient qu'ils avaient vu les îliens avec des plaques d'argent dans les oreilles et des colliers de perles autour du cou comme ornements. 
            
        
    Le soir, revenant aux navires \textit{Thienhoven} et \textit{Africaansche Galey} dans la baie, nous jetâmes une ancre par 22 brasses de fond, dans les récifs coralliens, prenant une orientation telle que la pointe est de l'île se trouvât à l'est-sud-est, la pointe ouest étant à l'ouest-nord-ouest\footnote{Ceci semble en accord avec la position de la baie de Lapérouse sur la côte nord de l'île.}.
            
        
    De nombreux canots venaient à nos navires\footnote{Dans son rapport du 8 avril, Brauman dit que « certains habitants de l'île venaient en bateau, d'autres venaient en nageant, sur des flotteurs faits de roseaux »}.
            
        
    À notre étonnement, ces gens montraient de l'intérêt pour tout ce qu'ils voyaient et, peu timides, ils prenaient les chapeaux et les bonnets sur la tête des marins et sautaient par-dessus bord avec leurs larcins. Ils étaient de bons nageurs étant donné qu'ils venaient de l'île à nos navires en nageant ; l'un d'entre eux sauta de son canot, par la fenêtre, dans la cabine du capitaine et voyant une nappe sur la table, jugeant que c'était une bonne prise, s'en empara et partit avec. Après ce larcin, une surveillance toute particulière fut nécessaire afin de conserver chaque chose en bon état.
            
        
    Nous préparâmes un débarquement avec 134 hommes afin de faire des observations sur l'île pour nos rapports.
            
        
    Le matin suivant, nous partîmes avec trois bateaux, deux chaloupes et cent trente-quatre hommes bien armés de fusils, de sacs de cartouches et d'épées.
            
        
    En arrivant à la côte, nous mîmes les bateaux et les chaloupes à l'ancre en prévoyant une garde de 20 hommes armés pour leur sécurité. Le bateau \textit{Africaansche Galey} était également équipé de deux petits canons\footnote{Bassen : petit canon.}.
            
        
    Tout ceci étant réglé, nous marchâmes groupés et non pas en rang, à cause des nombreux rochers sur le rivage. En arrivant dans la plaine, nous vîmes des habitants en grand nombre, venant vers nous.
            
        
    Aussitôt, les marins des trois bateaux étaient mis en ordre de bataille sous les ordres des capitaines Koster, Bouman et Rosendaal, formant trois rangées l'une derrière l'autre, ces marins étant protégés par la moitié des soldats sous les ordres de l'enseigne Martinus Keerens. Après application de ces ordres, nous allâmes un peu de l'avant afin de donner plus de place à nos gens qui étaient en arrière. A notre grand étonnement et avec surprise nous entendîmes quatre à cinq coups de mousquets tirés derrière nous, avec des cris très forts : « tirez, tirez\footnote{Bauman dit que « ce rapport par son sous-maître Cornélius était inexact. Roggeveen, Koster, Rosendahl, Keerens, lui-même, tous les officiers et de nombreux hommes, continuaient à dire que son rapport était inexact en dépit de nombreux hommes qui continuaient à dire qu'il était correct ».} ! ».
            
        
    Instantanément plus de trente mousquets tirèrent, les indiens étant complètement surpris et effrayés laissèrent derrière eux dix à douze morts, en plus des blessés.
            
        
    Les éléments en tête de l'expédition s'arrêtèrent et demandèrent qui avait donné l'ordre de tirer, et pour quelle raison. Peu de temps après un officier du bateau \textit{Thienhoven} expliqua qu'il était à l'arrière et qu'un des habitants de l'île avait saisi le canon de son fusil afin de le lui prendre de force, pendant qu'un autre indien le tirait par sa blouse. Certains habitants de l'île voyant notre résistance, ramassèrent des pierres avec des gestes menaçants. De toute évidence le tir de ma petite troupe était dû à cette menace, sans qu'aucun ordre de tirer n'ait été donné. Ce n'était pas le moment de rechercher des informations, il était préférable de remettre cela à plus tard lors d'une meilleure opportunité. L'étonnement et la peur des habitants les avaient rendus craintifs, puis ils se rendirent compte que nous ne continuions pas les hostilités et nous leur avons fait comprendre par gestes que la mort était proche s'ils continuaient à nous menacer avec des pierres.
            
        
    Les habitants de l'île qui étaient prés de nous et qui faisaient face revenaient vers le chef des officiers et l'un d'entre eux qui semblait avoir de l'autorité sur les autres, donna des ordres afin que l'on nous apportât des fruits, des légumes, des volailles… et s'inclina, témoignant de son regret de cet incident. Peu de temps après, ils apportèrent une grande quantité de canne à sucre, de volailles, d'igname, de bananes et nous leur fîmes comprendre par gestes que nous ne voulions rien, excepté les volailles, environ 60 et que nous avions largement payées — ainsi que 30 régimes de bananes — avec des tissus rayés qui leur plaisaient.
            
        
    Après avoir remarqué les colifichets qu'ils portaient, leur forme et leur couleur, nous qui avions imaginé voir des plaques d'argent et des colliers de perles, nous constatâmes que ces objets étaient sans valeur. Les vêtements qu'ils portaient, simples et de bon goût étaient faits de morceaux préparés à partir d'une plante cultivée dans leurs champs et cousus en trois ou quatre épaisseurs\footnote{Les habits, au temps de Roggeveen, comme partout en Polynésie aux temps anciens, étaient confectionnés à partir de l'écorce du murier (Broussonetia payrifera) : Alfred Métraux, « L'Ethnologie de l'île de Pâques », Bernice P. Bishop Museum Bulletin 160 (1940), p. 213-160.}.
            
        
    La terre des champs était rouge et jaunâtre ; mélangée avec de l'eau elle permettait, après trempage et séchage, de teinter ces morceaux de vêtements. Cette teinture était légère et s'enlevait facilement, elle restait sur leurs doigts, simplement par le toucher sur les nouveaux habits, mais aussi sur ceux qui étaient anciens et usés\footnote{Turméric (Curcuma longa) était employé autrefois sur l'île comme pigment jaune ou orange pour teindre les habits : Métraux, op. cit., p.158 et 236. La teinture Turméric n'est pas très stable. Il n'est pas certain qu'il ait vu la terre employée pour teindre les habits. Dans les temps historiques les habitants de l'Ile de Pâques utilisaient un produit volcanique rouge-brun pour enduire leur corps (Métraux op. cit., p. 236 ; Geiseler, un lieutenant de marine allemand qui visitait l'île en 1882 remarqua qu'un pigment jaune était obtenu par les îliens à partir d'un trou fait dans un cratère volcanique : Geisler, Die oster-Insel, Berlin, 1883, p 151).}.
            
        
    Ce que nous avions pensé être des plaques d'argent accrochés aux oreilles étaient des morceaux ronds ou ovales de racines de panais ou de carottes. Le diamètre allait de 1,5 pouce pour les plus petites jusqu'à 3 pour les importantes. Il faut savoir que dans cette population, les jeunes ont le lobe de l'oreille étiré, une petite partie est fendue et la rondelle blanche est insérée dans cette ouverture puis poussée vers la partie la plus large en la fermant\footnote{Le percement des lobes de l'oreille et l'agrandissement de ce trou par l'insertion de feuilles de canne à sucre roulées ou autres objets se font comme dans les temps anciens. Les plaques ornant les oreilles faites à partir de plantes ne sont pas les ornements les plus couramment remarqués, il y a des vertèbres de requin, des morceaux de bois ou d'os ; Métraux , op. cit., pp 228-9,235. Les plaques citées par Roggeveen peuvent avoir été coupées à partir de racines de taro.}. Les perles qui ornaient le cou de ces gens provenaient de coquillages plats dont la couleur intérieure était semblable à celle de nos coquilles d'huîtres\footnote{Geiseler, op. cit., p. 49, cite un ornement de cou de coquille de moule.}.
            
        
    Quand ces indiens étaient aux travaux des champs ou nageaient, ces pendants d'oreilles étaient incommodes. Ils les retiraient et relevaient le lobe de l'oreille vers le haut, ce qui leur donnait une étrange apparence comique.
            
        
    Ces gens ont un corps bien proportionné, d'assez grande taille, paraissant vigoureux et bien musclés. Leur couleur naturelle n'est pas noire, mais jaune ou jaunâtre. Nous vîmes de nombreux jeunes, parmi lesquels certains n'avaient pas de peinture bleue sur le corps\footnote{La peinture bleue foncée sur les corps était sans aucun doute des tatouages fait par incision de la peau et insertion de ce pigment bleu. Métraux, op. cit., p. 237-48. Roggeeven suggérait que la couleur de la peau à des degrés divers était due à une exposition au soleil plus ou moins longue. Plus tard, Cook donnait en référence les Tahitiens : J. Cook, The Journals of Captain James Cook, ed. J. C. Beaglehole, I, Cambridge.} parce qu'ils étaient d'un rang élevé et ne devaient pas participer aux travaux des champs.
            
        
    Ces gens ont les dents blanches comme la neige et une bonne dentition. Même les vieilles personnes aux cheveux gris que nous pouvions observer croquaient de larges coquilles dures, aussi épaisses que nos noyaux de pêches. Dans l'ensemble, leurs cheveux et la barbe étaient courts et de couleur claire. D'autres avaient les cheveux longs, tombant dans le dos ou en touffe sur le dessus de la tête, comme certains chinois de Batavia. 
            
        
    Nous ne connaissions que peu de choses de la religion de ces gens, notre séjour étant trop court. Nous observions qu'ils faisaient des feux devant de grandes statues de pierre et qu'ils s'asseyaient sur leurs talons, penchant la tête et levant et abaissant leurs mains\footnote{Behrens, op. cit., I , p. 125 et 135, cite après avoir vu du Den Arend ancré dans la baie de Lapérouse, le 8 avril, que les îliens étaient observés à la lueur de leurs feux aux pieds des idoles faisant des offrandes et les implorants. Le matin suivant, des adorateurs prostrés aux mêmes endroits, devant plusieurs feux, la figure regardant le soleil, honoraient leurs idoles. Certains îliens en prière paraissaient avoir plus de dévotion et de zèle, ils étaient certainement aux service des idoles, la plupart avaient des attributs distinctifs posés sur leur tête rasée, faits de plumes blanches et noires ressemblant à des cigognes. La suggestion qu'ils étaient en prière est donnée par le témoignage de Francisco Antonio de Agüera y Infanzon, chef pilote d'un des bateaux espagnols de l'expédition commandée par Félipe Gonzaléz de Haedo, qui était la dernière expédition européenne connue à avoir visitée l'île de Pâques en 1770. Agüera conclut qu'ils avaient des prêtres attachés aux idoles et qu'ils vivaient dans des résidences près des statues : The Voyage of Captain Don Felipe Gonzalez…..to Easter Island, 1770, I, ed. B . G Corney, Cambridge 1908, p. 100, 102 ; une traduction par Corney d'un passage du Journal de Roggeveen of the Easter Island, p. 3- 25.}. Ces grandes statues de pierre nous étonnèrent. Nous ne pouvions comprendre comment ces gens dépourvus de grosses poutres de bois pour fabriquer quelques dispositifs, et de même dépourvus de forts cordages, avaient pu ériger ces statues, lesquelles avaient plus de 30 pieds de hauteur avec une épaisseur en proportion. Mais notre étonnement cessa avec la découverte d'un morceau de pierre fait d'argile ou de terre glaise dont ces statues étaient faites. Ces statues ayant une apparence humaine étaient rangées ensemble avec ordre. Elles présentaient un léger relief descendant des épaules jusqu'aux pieds, figurant des bras. De même était pendu autour du cou, un long vêtement descendant jusqu'au sol. Sur la tête un panier dans lequel étaient posées des pierres blanches\footnote{Les grandes statues de l'Ile de Pâques n'étaient pas faites d'argile ni de terre glaise mélangés avec des galets, mais faites à partir de la pierre du volcan éteint le Rano Raraku. L'archéologue de l'expédition conduite par Thor Heyerdahl, lequel resta dans l'île du 27 octobre 1955 au 6 avril 1956, menait une étude importante sur les statues. : Reports of the norwégian Archaeological Expédition to Easter Island, and the East Pacific, ed. T. Heyerdahl et E.N Ferdon, I, Archaeology of Easter Island, Moographie de la "School of América Reseach" et le Muséum de New Mexico, n° 24, Partie I (1961). L'expédition démontra qu'il était possible de transporter et d'ériger les statues sur les Ahu avec la main d'œuvre et le matériel existant localement. Les statues n'ont pas de pieds et se terminent par une base plate sous le torse. Le panier observé par Roggeveen est une pierre posée sur la tête ; ceci correspond à une vieille tradition des temps anciens ou des galets de corail étaient jetés sur les assistants, mais Agüera (Corney , op. cit., p. 93) dit que des os d'anciens décédés étaient placés dans de petites cavités à la partie supérieure. Les statues étaient érigées en monuments mortuaires dédiées à des chefs ou à des personnalités importantes, la face orientée vers l'intérieur de l'île et tournant le dos à la mer. Dans les temps anciens la plupart des statues ont été abattues délibérément ou sont usées par l'érosion. Plus de six-cents statues terminées et un nombre important de statues non terminées ont été trouvées prés de la carrière du Rano Raraku. Une illustration montre deux statues encore en position sur une plate forme quand l'explorateur français Lapérouse visitait l'île en 1786.}.
            
        
    Comme personne ne pouvait voir s'ils avaient des pots en terre, poêles ou autres récipients, nous ne comprenions pas comment ces gens cuisaient leur nourriture. Nous observâmes qu'ils creusaient des trous dans la terre avec leurs mains et posaient des pierres, puis ils apportaient des brindilles séchées, les posaient dessus et les enflammaient. Peu de temps après, ils nous apportèrent pour manger une volaille cuite enveloppée dans une sorte de joncs, alléchante, de bel aspect blanc et chaud ; nous les remerciâmes par signes parce que nous avions la tâche de surveiller nos gens afin de les garder bien disciplinés et de cette manière ils ne pouvaient pas nous causer le moindre mal mais aussi en cas de désordre, ne pas être surpris. Quoique ces gens nous témoignèrent des marques d'amitié, l'expérience nous avait appris qu'il était préférable d'avoir une certaine méfiance. Comme le rapporte le journal de la flotte de Nasseau-Fleet, dix-sept hommes avaient été tués sur la Terre de Feu, trompés alors qu'ils rendaient de loyaux et bons services aux habitants. Nous pouvions alors observer avec attention et conclure qu'il devait y avoir dans la roche un creux renfermant de l'eau, laquelle permettait la cuisson des aliments. Ils posaient alors des pierres sur lesquelles ils installaient le foyer. La chaleur dégagée cuisait les aliments comme ils le souhaitaient, tendre ou plus cuits\footnote{La description du four par Roggeveen recoupait la supposition qu'il y avait de grands trous renfermant de l'eau sous la terre ; ce genre de four est généralement utilisé en Polynésie. Bouman en donne une description. Un feu est fait sur les pierres posées sur le sol, jusqu'à ce qu'elles soient chaudes, la nourriture enveloppée dans des feuilles ou des joncs est posée dessus, puis recouverte de terre ; Métraux, op. cit., p. 162. Behrens, op. cit., I, p. 131, cite les insulaires qui préparent leur nourriture en utilisant des pots en terre, sans aucun doute preuve d'imagination. Bouman mentionne qu'il y avait également des calebasses contenant de l'eau, qu'il la goûta mais la trouva saumâtre.}.
            
        
    Il est à remarquer que nous ne vîmes pas plus de deux à trois vieilles femmes qui portaient un vêtement allant de la taille au dessous des genoux et un autre autour des épaules, de telle manière que la poitrine était dénudée. Les jeunes femmes et les jeunes filles ne se montraient pas, on pouvait penser que la jalousie avait incité les hommes à les cacher dans une autre partie de l'île\footnote{Ceci est différent de ce que dit Behrens (op. cit., I, p. 134). « Les femmes s'asseyaient souvent près d'eux, se déshabillaient en riant et les séduisaient par toute sorte de gestes, pendant que d'autres appelaient les visiteurs et les faisaient venir dans leur maison ». Baumann fait mention de tels agissements. Aguera (Corney, op. cit., p. 97) dit que, quand il visita l'île, « les femmes courtisaient les visiteurs, encouragées par les hommes ».}.
            
        
    Leurs habitations ou huttes étaient sans ornement. Nous estimions qu'elles avaient une longueur de 50 pieds, une largeur de 15 et une hauteur de 9. Comme nous vîmes une hutte en construction, la structure de celle-ci nous apparut. Les perches de bois constituant les murs sont d'abord posés sur le sol attachées entre elles et liées à de longues pièces de bois de 4 à 5 pieds de haut. Les vides sont fermés avec une sorte de rideau fait de joncs ou de longues herbes épaisses attachées à une poutre en bois et par des cordes. Ils savaient comment faire ces nattes soigneusement et avec habileté, à partir d'une plante appelée « piet\footnote{Ici Roggeveen emploie le nom indien pour le murier à partir duquel, les îliens faisaient leurs vêtements. Plus tard ils voyaient que des cordes tressées de même fibre étaient utilisées pour lier les poutres de leur canot. C'est un document historique qui révèle que l'arbre mûrier servait autrefois dans l'Ile de Pâques pour fabriquer les cordages dont ils avaient besoin. Ces fibres sont également utilisées pour lier les bois dans la fabrication des barques. Il est confirmé qu'autrefois dans l'île de Pâques, les fibres de cet arbre étaient utilisées pour faire des cordages : Métraux, op. cit., p 21.} ». Avec cette couverture, ils sont bien protégés contre le vent et la pluie, comme les Hollandais vivant dans leur maison au toit de chaume. Ces huttes n'avaient qu'une entrée si basse que l'on ne pouvait y pénétrer qu'en rampant sur les genoux. Elle était ronde avec, au dessus, une voûte. L'aménagement intérieur que nous pouvions voir difficilement, la hutte étant sans fenêtre et l'intérieur sombre, laissait voir une natte sur le sol et de grandes pierres servant vraisemblablement d'oreillers. En outre, il y avait à l'intérieur de la hutte de grandes pierres de 3 à 4 pieds de largeur, posées les unes à côté des autres régulièrement, et qui, à notre avis servaient de sièges lors des conversations dans le calme du soir. Et pour terminer à propos de ces huttes, nous ne vîmes sur le terrain où nous étions que six à sept huttes ; nous pouvions en conclure que ces îliens faisaient un usage commun de ce qu'ils possédaient. Les dimensions et le petit nombre de huttes révélaient que beaucoup vivaient et dormaient ensemble dans une grande hutte, mais, partant de cette observation, en déduire le partage des femmes entre eux serait une accusation légère et diffamatoire\footnote{Bouman dit que « certaines huttes ressemblaient à des ruches et d'autres à des bateaux nordiques retournés (kayaks), le dessous devenant le dessus de la hutte. L'usage de baliveau et de chaume assemblés en forme de bateaux retournés, avec une petite entrée basse et à l'intérieur des pierres comme oreillers est confirmé plus tard par l'histoire locale. « Habituellement, pour vivre dehors, il existe devant les maisons une petite cour pavée » : Métraux, op. cit., p. 210.}.
            
        
    En regardant leurs bateaux, nous les trouvions fragiles étant donné l'usage qu'ils en faisaient. Leurs petits canots sont faits de petites planches avec à l'intérieur de légères poutres liées ensemble avec des fils torsadés faits de la plante déjà nommée, « piet ». Mais comme ils ne connaissaient pas les matériaux pour calfater leurs barques, ils étaient obligés de faire un grand nombre de coutures pour imperméabiliser la coque, les rendant inutilisable pendant un temps assez long. Les canots ont 10 pieds de longueur et une proue pointue ; leur largeur est telle qu'ils peuvent juste s'asseoir à l'avant pour pagayer.
            
        
    Le roi ou le chef nous invita à aller de l'autre coté de l'île où se trouvaient leurs terres cultivées et les arbres fruitiers d'où provenaient ce qu'ils nous apportaient. Cette proposition fut jugée imprudente, le vent du nord commençant à souffler. L'ancrage de nos bateaux était à surveiller et nous n'avions que peu de personnel à bord qui aurait besoin d'aide si le vent devenait plus violent. En outre, les canots et les chaloupes étaient remplis de membres d'équipage qui n'auraient pu atteindre les navires à cause des brisants près des côtes et l'impossibilité à ramer. Nous avons donc décidé de ramener nos gens rapidement.
            
        
    Nous décidâmes de naviguer environ 100 miles vers l'ouest, en faisant un léger détour vers l'est, afin d'observer à nouveau la côte basse et sablonneuse que nous avions découverte lors de notre arrivée. Notre première navigation dans les mers du sud ayant atteint son but, nous devions nécessairement faire le rapport de ce que nous avions découvert. Quand il fut décidé d'aller dans l'île, les chefs des trois bateaux se sont réunis à bord du Den Arden et ont proposé d'utiliser trois bateaux et des chaloupes avec des hommes armés.
         
         
            
        \section*{Vendredi 10 avril 1722}
    
            
        
    Le commandant ayant rassemblé les chefs et les responsables de l'expédition, communiqua ses observations et donna son opinion concernant cette nouvelle île découverte. Un second objectif fut admis et agréé par le conseil : l'île serait observée et décrite, quoique cela fut contraire au programme de notre expédition.
            
        
    Elle était éloignée de 100 miles environ par rapport à nos prévisions. Cette terre découverte le jour de Pâques ne peut être qualifiée de basse et sableuse, comme l'\textit{Africaansche Galey} le signala quand elle était distante de 8 à 9 miles. 
            
        
    A ce moment nous décidâmes de remettre au lendemain notre approche de l'île. Le jour suivant, avec une légère brise par le travers, nous approchâmes à environ 2 miles. Cette terre ne pouvait être qualifiée de sableuse mais, au contraire, exceptionnellement féconde, produisant des fruits, des bananes, des patates douces, de la canne à sucre et d'autres produits, en dépit de l'absence de grands arbres et de bétail, excepté de la volaille\footnote{Bouman ajoute que « les îliens cultivaient soigneusement des carrés de terre, qu'ils avaient des cocotiers, et qu'ils employaient des petits couteaux en pierre (sans doute en obsidienne) pour couper les bananes du bananier ».}.
            
        
    Ainsi, cette île, avec ses terres fertiles et son climat agréable, pouvait être considérée comme un paradis terrestre, si elle était convenablement cultivée et travaillée, alors qu'à présent, les habitants se satisfont de ce qu'ils jugent nécessaire à leur survie. De plus, il est complètement faux de décrire cette terre découverte comme une rangée de hautes terres, même si on peut supposer que l'on était par malchance passé à coté de l'île sableuse sans la voir.
            
        
    Notre programme de navigation fut ainsi décidé : nous devions voir inévitablement si cette Ile de Pâques était cette terre décrite comme une rangée de hautes terres. Donc nous pouvions penser, avec de bonnes raisons, que cette Ile de Pâques était une autre terre que celle que nous recherchions et qu'elle pouvait constituer une partie de notre programme. Ainsi, le président donna à considérer au conseil, tous les points ci-dessus, afin que son point de vue fut convenablement admis.
            
        
    Ceci étant réglé et approuvé, il était indiscutable que l'information connue comme indiquant de hautes montagnes sur l'île de Pâques n'était pas exacte, ces dernières n'étant que d'une hauteur moyenne. Egalement il n'y avait pas présence de métal, ce que nous avions constaté. De même, pour se couvrir, les habitants employaient une plante qu'ils savaient coudre en trois ou quatre épaisseurs d'une manière élégante pour avoir chaud. En outre, les femmes portaient comme ornement sur leur tête une guirlande faite de plumes de volailles ou d'oiseaux (ces plumes d'oiseaux n'étant vues que très rarement). Leur visage était peint tout comme d'autres parties du corps, avec des dessins d'une conformation régulière. Elles avaient également des coquilles plates comme ornements de nez, et pour décorer les oreilles qui étaient percées, une sorte de racine ressemblant au panais de chez nous\footnote{Il est généralement admis parmi les linguistes et anthropologues spécialisés dans les études concernant le Pacifique que le langage et la culture de l'Ile de Pâques dans les temps anciens étaient reliés aux autres habitants de la Polynésie ; e.g.S.H. Elbert, « Internal Relationships of Polynéésian Languages et dialects » Southwestern journal of Anthropology, IX.147-73 ; R. Green, "Linguistic Subgrouping within Polynesia : The implacations for Prehistoric Settlement",journal of the Polynésian Society, Ixxv.6-38; A. Pawley, Polynesia Languages; Un Sougroupe basé sur innovations partagées' Journal of the Polynesian Society, XXV.39 64 ; E. G .Burrows'Western Polynesia. Une étude de différenciation des cultures Etnologiska Studier,,VII- 1-192 ; Métraux, op. cit. Observations ethnologiques de Roggeveen et de ses compagnons au moment de leur visite à l'Ile de Pâques, voyaient que la culture était principalement polynésienne, mais pas exclusivement. Heyerdahl et Ferdon dans leurs observations générales pensent que Bouman prend comme référence qu'autrefois certains îliens nageaient sur des radeaux de roseaux, Reed signale des dessins de radeaux avec sur les voiles des peintures et admettait que la culture de l'île au moment de leur visite était bien marquée, mais non exclusivement de caractère Polynésien. Heyerdahl et Ferdon dans leurs travaux (op. cit. i. 493-526, 527- 35 ) ne doutaient pas que le langage et la culture d'autrefois, archéologique et botanique avaient une ascendance polynésienne, que certains signes repérés dans les temps anciens provenant d'Amérique du sud, étaient possibles. Roggeveen et Bouman ont fait certaines observations dans l'île qui tenteraient à induire une influence sud-Américaine. La patate douce appelée pataddes par Roggeveen était indubitablement d'origine américaine, son usage était largement répandu en Polynésie, elle pouvait venir de nombreux endroits de ces îles, les Polynésiens l'ayant léguée à leurs descendants. Bauman note le fait que certains habitants de l'île utilisaient autrefois des radeaux faits de roseaux, la présence de ces radeaux est confirmée par des peintures et des figures (pétroglyphes) gravées sur des pierres ; Ferdon, dans Heyerdahl et Ferdon op.cit. i. 534-5 signale une corrélation forte de ces cultures polynésiennes avec celles existantes en Amérique du sud. Il y a une forte évidence de contacts entre ces populations.}.
            
        
    Par ailleurs, nous n'avons pas vu la petite côte basse et sablonneuse qui devait être le signe avant-coureur de ce que nous recherchions. 
            
        
    Donc, ce fut par consentement mutuel et approuvé : nous continuerons notre course vers l'ouest, à la latitude sud de 27 degrés, jusqu'à ce que nous ayons navigué et atteint une centaine de miles. 
            
        
    Le jour même, nous préparâmes le programme de ce que nous devions faire, puis celui-ci fut admis et signé par Jacob Roggeveen, Jan Koster, Cornélius Bouman et Roclof Rosendaal. Ce programme étant adopté, le capitaine Jan Koster suggéra qu'il serait facile d'étudier, avec les éléments déjà mentionnés sur l'Ile de Pâques, si elle était réellement la terre que nous recherchions et vers laquelle nous avions dirigé notre navigation. Si nous entreprenions simplement un petit voyage de 12 milles vers l'est, les bateaux étant distants de 2 miles l'un de l'autre, nous pourrions alors avoir un succès certain si nous découvrions une île basse et sableuse. En vérité, cette terre déjà nommée Ile de Pâques est une autre île que celle prévue initialement dans notre programme de navigation.
            
        
    Avec un vent plus soutenu et plus fort, si nous découvrions une île sableuse, nous avancerions dans la réalisation de notre seconde entreprise.
            
        
    Tout cela bien considéré, nous avons adopté la résolution ci-dessus et le même jour, elle fut signée par Jacob Roggeveen, Jan Koster, Cornélius Bouman, Roelof Rosendaal.
         
      
   
        \end{document}