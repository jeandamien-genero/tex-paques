\documentclass{article}
        \usepackage[utf8]{inputenc}
        \usepackage{xcolor}
        \usepackage[colorlinks=true, allcolors=darkgray]{hyperref}
        \usepackage{chngcntr}% http://ctan.org/pkg/chngcntr
        \counterwithout{subsubsection}{subsection}
        \renewcommand\thesubsection{\Roman{subsection}.}
        \renewcommand\thesubsubsection{§ \arabic{subsubsection}}
        \renewcommand*\contentsname{\centering{Sommaire}}
        
        \title{Lettre du R. P. Barnabé au T.-R. P. Supérieur général\footnote{Ce texte, originellement publié dans \textit{Annales de la propagation de la foi} n°38, Lyon, 1866, pp.139-145, a été encodé dans un document XML puis transformé en fichier \LaTeX{} grâce à une feuille de transformation XSL, par Jean-Damien Généro, ingénieur d'études du CNRS affecté au Centre de recherches historiques (UMR 8558/ CNRS-EHESS).}}
        \author{Barnabé Castan}
        \date{Valparaiso, 9 février 1865}
        
        \begin{document}
        \maketitle
        
      
         
        
    \href{https://gallica.bnf.fr/iiif/ark:/12148/bpt6k5449280k/f146/full/full/0/native.jpg}{\textbf{[139]}} « Mon Très-révérend Père,
         
        
    « Le 25 août dernier, fête de saint Louis, roi de France, le F. Hugues et moi, nous quittâmes Valparaiso pour nous rendre à l'île de Pâques, et voir ce qu'était devenu le F. Eugène. Nous étions à bord de la \textit{Térésa-Ramos}, goélette chilienne, qui d'abord allait porter des vivres aux mineurs de Tres-Puntas, dans la province d'Ataeama. Ce petit détour nous retarda jusqu'au 14 septembre, jour où nous primes définitivement la route de l'île de Pâques.
         
        
    « A peine avons-nous perdu de vue la terre, que le vent se met à souffler avec violence. La mer blanchit d'écume et couvre notre petit navire, et un froid très-vif se fait sentir. Notre capitaine tombe malade, et peut difficilement commander les trois matelots qui composent l'équipage ; et enfin, le chronomètre s'étant arrêté, nous ne savons plus à quelle hauteur nous nous trouvons. Nous voguons ainsi vingt-quatre jours, à l'aventure et à la merci des vents. Nos yeux étaient fatigués d'interroger vainement tous les points de l'horizon. Cependant, le 3 octobre, l'apparition d'un trois mâts, au nord, nous apporta quelque soulagement. Aussitôt le capitaine arbore le pavillon de détresse, et pendant trois heures nous courons sur ce navire y sans pouvoir nous en approcher, ni même obtenir de lui un signe de salut amical. Il fuyait devant nous à \href{https://gallica.bnf.fr/iiif/ark:/12148/bpt6k5449280k/f147/full/full/0/native.jpg}{\textbf{[140]}} toutes voiles. Il est à croire qu'il nous prenait pour des pirates, ou qu'il en portait lui-même. Le dimanche suivant, 9 octobre, le capitaine nous prévient que, le lendemain à midi, il prendra sa dernière hauteur, et que, si l'île n'apparaît point, il retournera en arrière pour la chercher, jusqu'à ce qu'il n'ait plus d'eau à bord. Nous étions tous dans une grande inquiétude. J'avais pourtant un motif de consolation : c'est que j'avais entrepris ce voyage par obéissance ; je me mis à prier et à faire au bon Dieu, s'il le désirait, le sacrifice de ma vie.
         
        
    « Mais voilà que le lendemain 10 octobre, à onze heures du matin, tout change de face. Le capitaine, qui était monté à la mâture avec la longue-vue, redescend tout joyeux : cette fois il a vu la terre, et le navire allait dessus, vent arrière. Nous ne tardons pas à apercevoir l'île de Pâques, et ses mamelons, et ses effrayantes falaises. A mesure que nous approchons, le vent fraîchit, de bien loin nous voyons la mer se briser et s'élever comme de la fumée à une hauteur prodigieuse ; toute la partie orientale de l'île parait battue par une tempête continuelle. Nous cinglons vers le sud, pour passer la nuit au large.
         
        
    « Le lendemain, de bonne heure, nous sommes vus par les insulaires. Le capitaine se dirige vers le port d'Anakena, et, chemin faisant, nous apercevons, au fond d'une petite baie, une case européenne. Le F. Hugues et moi nous prions le capitaine d'approcher davantage ; mais le vent d'est, qui soufflait fort et nous rejetait au large, ne nous laissait guère l'espoir d'entrer dans la baie d'Anarova. Trois ou quatre heures de bordées \href{https://gallica.bnf.fr/iiif/ark:/12148/bpt6k5449280k/f148/full/full/0/native.jpg}{\textbf{[141]}} nous mirent à peine en communication avec la terre. Sur ces entrefaites, nous avions hissé le pavillon français. De la plage, qui s'était couverte d'indigènes, on nous répondit en hissant un pavillon blanc pour nous indiquer le mouillage, et nous annoncer une réception pacifique. Quand la goélette arrivait à une pointe de la baie pour virer de bord et s'approcher toujours davantage, les Kanacs, perchés sur les falaises, nous criaient tous ensemble;:
         
        
    « – \textit{Holo mai ! hoto mai !Viens ! viens !} »
         
        
    « Je leur répétai les mêmes paroles, en agitant mon mouchoir. Le capitaine me demanda si j'avais compris ce qu'ils disaient.
         
        
    « – Parfaitement, répondis-je ; c'est du pur sandwichois. »
         
        
    « Aussitôt nous voyons les Kanacs venir vers nous à la nage ; et nous n'avions pas encore mouillé qu'ils étaient à bord. Une femme monta la première, et, quoiqu'elle n'eût jamais vu aucun prêtre, elle s'adresse tout d'abord à moi, fait le signe de la croix, et récite le \textit{Pater}, l’\textit{Ave} et le \textit{Credo} en langue tahitienne. Pour l'encourager et pour voir si elle me comprendra, je me mets à réciter les mêmes prières en langue sandwichoise. Elle m'entend très-bien. D'autres Kanacs arrivent, et parmi eux Tamateka, un de ceux qui ont échappé aux pirates du Pérou. Il me parle des Pères et des Frères qu'il a vus à Tahiti et aux Gambiers. En voyant leur empressement à réciter les prières, je demande qui leur a enseigné ces paroles si belles et si bonnes.
         
        
    « – \textit{Le Papa, le Papa}\footnote{L'étranger.}, » répondent-ils tous d'une voix.
         
        
    « – Et où est-il ce \textit{Papa} ? » leur dis-je.
         
        
    « Ils me montrent sa maison : c'était celle-là même \href{https://gallica.bnf.fr/iiif/ark:/12148/bpt6k5449280k/f149/full/full/0/native.jpg}{\textbf{[142]}} qui avait fixé notre attention. Quant au \textit{Papa}, d'après les expressions enthousiastes avec, lesquelles nos insulaires le dépeignaient, j'étais presque sûr qu'il était vivant. Cependant un mot. mal compris par eux, ou n'ayant pas la signification que j'y attachais, faillit me compromettre. Je leur demandai en sandwichois s'ils aimaient beaucoup le Papa qui les instruisait. Il paraît que le mot dont je me servis signifie tuer, à l'île de Pâques. La réponse fut un non énergique, dit sur un ton qui exprimait bien l'horreur des Kanacs pour l'effusion du sang »
         
        
    « Je ne dois pas oublier d'ajouter, mon très-révérend Père, que les Kanacs, après avoir récité les prières, s'informèrent du nom du capitaine, de sa nationalité, d'où nous venions, s'il y avait à bord des gens du Callao. Le nom du Callao les fait frissonner de peur. Il paraît que, si le F. Eugène ne les avait pas rassurés quand il vit le pavillon français, ils nous auraient reçus à coups de pierres et à coups de lances.
         
        
    « Nous nous disposons à descendre à terre, le F. Hugues et moi. Mais, dès que nous sommes dans le canot, nos Kanacs l'envahissent, appelant même ceux qui nagent encore autour du navire. Les matelots ne peuvent ramer, et nous allons couler ou tomber sur les récifs. Tous les sauvages, que la nouveauté du spectacle égaie extrêmement, crient à tue-tête, sans qu'on puisse les faire taire. Enfin, lorsqu'ils sont fatigués de notre compagnie, ils nous débarrassent de leur présence, emportant tout ce qu'ils ont pu attraper dans le canot. A mesure que nous approchons, nous distinguons parfaitement une foule compacte et silencieuse d'hommes armés de lances et de bâtons. Leur accoutrement, leur figure bariolée, leur \href{https://gallica.bnf.fr/iiif/ark:/12148/bpt6k5449280k/f150/full/full/0/native.jpg}{\textbf{[143]}} contenance, tout serait capable d'épouvanter des gens moins hardis que nous.
         
        
    « — J'aperçois bien un Européen, dit le F. Hugues mais ce n'est pas le F. Eugène. »
         
        
    « Je me lève et fais signe aux Kanacs de venir nous prendre, pour nous empêcher de mouiller nos chaussures dans l'eau de mer; En réponse à mon appel, ils m'apportent cet Européen dont l'accoutrement bizarre nous avait causé plus de défiance que celui des autres Kanacs ; et c'est seulement quand il se jette à mon cou, que je reconnais le F. Eugène. Nous avions eu à peine le temps de nous embrasser, que les matelots, sans ordre ni signal, nous ramènent à bord.
         
        
    « Il était trois heures du soir, et le F. Eugène était à peu près à jeun ; il avait traversé l'île à pied pour venir nous voir. Notre premier soin, sur le navire, fut de lui donner de la nourriture et aussi des vêtements, car il arrivait presque nu.
         
        
    « Lorsqu'il m'eut raconté en gros ses aventures, je lui dis que j'étais envoyé pour savoir de ses nouvelles, que le mieux, à mon avis, serait de revenir avec moi à Valparaiso ; il n'était pas certain qu'on envoyât immédiatement des missionnaires à l'île de Pâques, et sa présence à Valparaiso pourrait être très-utile à ceux qu'on enverrait plus tard.
         
        
    « Le F. Eugène ne se souciait guère de quitter son île ; il regrettait les Kanacs, et nous nous apercevions bien que les Kanacs le regrettaient aussi. Il n'avait rien à prendre, la clef même de sa case était perdue. De son côté, le capitaine n'avait qu'un seul canot et peu de monde à bord ; il ne voulait pas exposer une seconde fois ce canot à être brisé ou volé par les Kanacs. Nous allons donc trouver Torometi et \href{https://gallica.bnf.fr/iiif/ark:/12148/bpt6k5449280k/f151/full/full/0/native.jpg}{\textbf{[144]}} plusieurs autres insulaires qui attendaient sur le pont. Nous leur annonçons que le Papa va à Valparaiso chercher deux prêtres, qui l'aideront à enseigner les prières et les cantiques aux enfants de \textit{Rapa-Nui}\footnote{L'île de Pâques.}, et nous recommandons à Torometi d'avoir soin de la case du F. Eugène jusqu'à son retour. Le discours se termina par une distribution de chemises. Le capitaine consent à faire reconduire les Kanacs au rivage, mais, cette fois, le pilote qui commande l'expédition s'est muni d'un revolver à six coups. Il l'a fait contre mon intention formelle. J'avais protesté, lorsque le capitaine préparait ses armes, en lui disant que nous ne venions pas faire la mission a coups de fusil, mais la croix à la main. Le capitaine avait cru, par prudence, ne pas devoir tenir compte de mes observations.
         
        
    « Arrivés à terre, Torometi et ses compagnons annoncent le départ du Papa. A cette nouvelle succède un morne silence ; chacun se retire, et, quelques minutes après, il ne reste plus personne sur la plage. Le capitaine ordonne de lever l'ancre, et nous repartons, quatre heures seulement après notre arrivée.
         
        
    « En abordant cette ile, je l'ai bénie au nom de la Congrégation, à l'exemple de notre vénérable fondateur, dont la main s'étendait pour bénir, en même temps qu'il terminait sa pieuse formule par ces mots : \textit{Et super insulas istas}\footnote{Et sur ces îles}.
         
        
    « Notre retour au Chili n'offre aucune particularité remarquable. Par la grâce de Dieu et les prières de nos amis, nous mouillâmes sans accident, le 30 octobre 1864, dans la rivière du Maule.
         
        
    « Si vous voulez savoir, mon très-révérend Père, quelle impression a faite sur moi la vue des habitants de l'île de Pâques, je vous répondrai que, si j'avais \href{https://gallica.bnf.fr/iiif/ark:/12148/bpt6k5449280k/f152/full/full/0/native.jpg}{\textbf{[145]}} été en compagnie d'un autre prêtre, je serais resté dans cette île, et que je suis disposé à partir avec un de nos Pères.
         
        
    « Veuillez agréer, mon très-révérend Père, l'expression de mon profond respect.
         
        
    « Barnabé Castan,
         
        
    « Prêtre des Sacrés-Coeurs
         
        
    « de Jésus et de Marie. »
      
  
        \end{document}